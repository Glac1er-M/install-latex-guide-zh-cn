% !TeX root = ../install-latex-guide-zh-cn.tex

\chapter{有关 Ubuntu 和 WSL 的补充}\label{chp:appendix:addition}

\section{更换 Ubuntu 源}\label{sec:addition:source}

鉴于大陆网络现状,
建议更换 Ubuntu 源至国内,
例如更换至清华大学.
在 \textsf{Terminal} 中执行
\begin{lstlisting}[language = bash]
  sudo cp /etc/apt/sources.list /etc/apt/sources.list.bak
\end{lstlisting}
备份 \texttt{sources.list} 文件.
接下来执行
\begin{lstlisting}[language = bash]
  sudo gedit /etc/apt/sources.list
\end{lstlisting}
将文件替换为以下内容%
\footnote{本段文字可至%
  \href{https://mirrors.tuna.tsinghua.edu.cn/help/ubuntu/}{清华大学镜像网站}%
  获取}
\begin{lstlisting}
  # 默认注释了源码镜像以提高 apt update 速度,如有需要可自行取消注释
  deb https://mirrors.tuna.tsinghua.edu.cn/ubuntu/ jammy main restricted universe multiverse
  # deb-src https://mirrors.tuna.tsinghua.edu.cn/ubuntu/ jammy main restricted universe multiverse
  deb https://mirrors.tuna.tsinghua.edu.cn/ubuntu/ jammy-updates main restricted universe multiverse
  # deb-src https://mirrors.tuna.tsinghua.edu.cn/ubuntu/ jammy-updates main restricted universe multiverse
  deb https://mirrors.tuna.tsinghua.edu.cn/ubuntu/ jammy-backports main restricted universe multiverse
  # deb-src https://mirrors.tuna.tsinghua.edu.cn/ubuntu/ jammy-backports main restricted universe multiverse

  # deb https://mirrors.tuna.tsinghua.edu.cn/ubuntu/ jammy-security main restricted universe multiverse
  # # deb-src https://mirrors.tuna.tsinghua.edu.cn/ubuntu/ jammy-security main restricted universe multiverse

  deb http://security.ubuntu.com/ubuntu/ jammy-security main restricted universe multiverse
  # deb-src http://security.ubuntu.com/ubuntu/ jammy-security main restricted universe multiverse

  # 预发布软件源,不建议启用
  # deb https://mirrors.tuna.tsinghua.edu.cn/ubuntu/ jammy-proposed main restricted universe multiverse
  # # deb-src https://mirrors.tuna.tsinghua.edu.cn/ubuntu/ jammy-proposed main restricted universe multiverse
\end{lstlisting}
保存退出.
再执行
\begin{lstlisting}[language=bash]
  sudo apt update && sudo apt upgrade
\end{lstlisting}
换源并更新.
若更改错误,
可执行
\begin{lstlisting}[language=bash]
  sudo cp /etc/apt/sources.list.bak /etc/apt/sources.list
\end{lstlisting}
恢复文件.

\section{安装字体}\label{sec:addition:font}

有用户指出,
Ubuntu 中安装字体比较麻烦.
这里引用
\href{https://www.jianshu.com/p/e7f12b8c8602}{Ubuntu系统字体命令和字体的安装}
一文.
大致步骤如下.

首先获取需要安装的字体文件,
假设文件保存在 \verb|~/fonts/|.
然后在 \texttt{/usr/share/fonts/} 文件夹中创建新的文件夹,
例如 \texttt{myfonts}
\begin{lstlisting}[language=bash]
  cd /usr/share/fonts/
  sudo mkdir myfonts
\end{lstlisting}
接下来将获取的字体文件复制到 \texttt{myfonts} 中
\begin{lstlisting}[language=bash]
  sudo cp ~/fonts/* /usr/share/fonts/myfonts/ 
\end{lstlisting}
然后修改字体文件的权限
\begin{lstlisting}[language=bash]
  sudo chmod -R 755 myfonts
\end{lstlisting}
最后建立字体缓存
\begin{lstlisting}[language=bash]
  mkfontscale
  mkfontdir
  fc-cache -fv
\end{lstlisting}

实际上有部分字体可直接使用源进行安装,
如安装字体 Noto CJK
\begin{lstlisting}[language=bash]
  sudo apt install fonts-noto-cjk
  sudo apt install fonts-noto-cjk-extra
\end{lstlisting}
安装 Windows 系统中常见的字体 Andale Mono,
Arial Black,
Arial (Bold, Italic, Bold Italic),
Comic Sans MS (Bold),
Courier New (Bold, Italic, Bold Italic),
Georgia (Bold, Italic, Bold Italic),
Impact,
Times New Roman (Bold, Italic, Bold Italic),
Trebuchet (Bold, Italic, Bold Italic),
Verdana (Bold, Italic, Bold Italic),
Webdings
\begin{lstlisting}[language=bash]
  sudo apt install ttf-mscorefonts-installer
\end{lstlisting}
安装字体文泉驿正黑,
文泉驿微米黑,
文泉驿等宽微米黑,
文泉驿点阵宋体,
文泉驿点阵楷体
\begin{lstlisting}[language=bash]
  sudo apt install ttf-wqy-zenhei
\end{lstlisting}

\section{反向代理 PPA}\label{sec:addition:proxy}

这里介绍中科大的\href{https://mirrors.ustc.edu.cn/}{反向代理},
将
\begin{lstlisting}[language = bash]
  /etc/apt/sources.list.d/
\end{lstlisting}
中文件的内容
\begin{lstlisting}
  http(s)://ppa.launchpad.net
\end{lstlisting}
改为
\begin{lstlisting}
  https://launchpad.proxy.ustclug.org
\end{lstlisting}
而后再
\begin{lstlisting}[language = bash]
  sudo apt update
\end{lstlisting}
即完成换源.
有时地址太多而逐个换太麻烦,
可以直接使用如下命令直接替换\footnote{参见 \url{https://mogeko.me/posts/zh-cn/035/}}
\begin{lstlisting}[language = bash]
  sudo find /etc/apt/sources.list.d/ -type f -name "*.list" -exec  sed  -i.bak -r  's#deb(-src)?\s*http(s)?://ppa.launchpad.net#deb\1 https\2://launchpad.proxy.ustclug.org#ig' {} \;
\end{lstlisting}

\section{使用 VS Code 的 WSL 插件编辑文本}\label{sec:vscode-wsl}

在 WSL 中编辑文本对于相当一部分 Windows 用户而言很不习惯,
因此这里给出另一种处理方法.
该方法借助 \href{https://code.visualstudio.com/}{VS Code} 的 \href{https://marketplace.visualstudio.com/items?itemName=ms-vscode-remote.remote-wsl}{WSL} 插件实现.
在 VS Code 中安装该插件即可.

更改 \texttt{sources.list}.
在 \textsf{bash} 中将 \texttt{sources.list} 备份为 \texttt{sources.list.bak}
\begin{lstlisting}[language=bash]
  sudo cp /etc/apt/sources.list /etc/apt/sources.list.bak
\end{lstlisting}
用 VS Code 打开文档
\begin{lstlisting}[language=bash]
  code /etc/apt/sources.list
\end{lstlisting}
并对其中内容进行更改,
并且另存为
\begin{lstlisting}
  ~/sources.list.sav
\end{lstlisting}
在 \textsf{bash} 中执行
\begin{lstlisting}[language=bash]
  sudo cp ~/sources.list.sav /etc/apt/sources.list
\end{lstlisting}
替换文档.
若发现文档有错误,
执行
\begin{lstlisting}[language=bash]
  sudo cp /etc/apt/sources.list.bak /etc/apt/sources.list
\end{lstlisting}
恢复文档.

更改 \texttt{.bashrc}.
在 \textsf{bash} 中将 \texttt{.bashrc} 备份为 \texttt{.bashrc.bak}
\begin{lstlisting}[language=bash]
  cp ~/.bashrc ~/.bashrc.bak
\end{lstlisting}
用 VS Code 打开文档
\begin{lstlisting}[language=bash]
  code ~/.bashrc
\end{lstlisting}
并对其中内容进行更改并保存.
若更改错误,
可以执行
\begin{lstlisting}[language=bash]
  cp ~/.bashrc.bak ~/.bashrc
\end{lstlisting}
恢复文档.

\section{用 VS Code 作为 WSL 中的编辑器}\label{sec:addition:wsl-editor}

这里介绍
\href{https://code.visualstudio.com/}{VS Code}
配合插件
\href{https://marketplace.visualstudio.com/items?itemName=ms-vscode-remote.remote-wsl}{WSL}
和
\href{https://marketplace.visualstudio.com/items?itemName=James-Yu.latex-workshop}{\LaTeX{} Workshop}
的方式.
以下配置为个人自用,
仅供参考,
其中注释的部分是调用外部 SumatraPDF 阅读器的配置,
用户可参考
\href{https://github.com/OsbertWang/latex-editor-sumatrapdf}{latex-editor-sumatrapdf}
来了解更多内容.

\begin{lstlisting}
  "latex-workshop.latex.tools": [
    {
      "name": "latexmkpdf",
      "command": "latexmk",
      "args": [
        "-synctex=1",
        "-interaction=nonstopmode",
        "-halt-on-error",
        "-file-line-error",
        "-pdf",
        "%DOCFILE%"
      ]
    },
    {
        "name": "latexmkxe",
        "command": "latexmk",
        "args": [
          "-synctex=1",
          "-interaction=nonstopmode",
          "-halt-on-error",
          "-file-line-error",
          "-xelatex",
          "%DOCFILE%"
        ]
      },
  ],
  "latex-workshop.latex.recipes": [
    {
      "name": "latexmkpdf",
      "tools": [
        "latexmkpdf"
      ]
    },
    {
        "name": "latexmkxe",
        "tools": [
          "latexmkxe"
        ]
      },
  ],
  "latex-workshop.latex.autoBuild.run": "never",
  "latex-workshop.view.pdf.viewer": "tab"
  // "latex-workshop.view.pdf.viewer": "external",
  // "latex-workshop.view.pdf.ref.viewer": "external",
  // "latex-workshop.view.pdf.external.viewer.command": "<SumatraPDFROOT>/SumatraPDF.exe",
  // "latex-workshop.view.pdf.external.viewer.args": [
  //   "-inverse-search",
  //   "\"<VSCodeROOT>/bin/code.cmd\" -r -g \"%f:%l\"",
  //   "%PDF%"
  // ],
  // "latex-workshop.view.pdf.external.synctex.command":"<SumatraPDFROOT>/SumatraPDF.exe",
  // "latex-workshop.view.pdf.external.synctex.args":[
  //   "-forward-search",
  //   "%TEX%",
  //   "%LINE%",
  //   "%PDF%",
  // ],
\end{lstlisting}

实际配置时,
用户需要结合自身安装情况更换 \texttt{<SumatraPDFROOT>} 和 \texttt{<VSCodeROOT>},
即 SumatraPDF 和 VS Code 各自的安装路径.