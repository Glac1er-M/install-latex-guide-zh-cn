% !TeX root = ../install-latex-guide-zh-cn.tex

\chapter{离线安装宏包}

由于一些原因,
有些电脑不能联网,
这对升级宏包不利.
这里介绍离线升级标准 \TeX{} Live 的包 (pkg) 的方法,
主要是用 \texttt{tlmgr install} 的 \texttt{--file} 参数来实现.

首先在能够联网的电脑上访问
\href{https://ctan.org/tex-archive/systems/texlive/tlnet/archive}{\texttt{archive}}
页面, 下载 \texttt{<archivename>.tar.xz} 文件.
文件列表很长, 加载需花费些时间.
注意 \texttt{<archivename>} 未必是 \href{https://ctan.org/pkg/}{pkg} 上的 \texttt{<pkgname>},
例如在 \texttt{pkg} 上的
\href{https://ctan.org/pkg/lshort-zh-cn}{lshort-zh-cn}
对应着 \texttt{archive} 上的
\href{http://mirrors.ctan.org/systems/texlive/tlnet/archive/lshort-chinese.tar.xz}{\texttt{lshort-chinese.tar.xz}}
和
\href{http://mirrors.ctan.org/systems/texlive/tlnet/archive/lshort-chinese.doc.tar.xz}{\texttt{lshort-chinese.doc.tar.xz}}.

这里多解释一点 \texttt{<archivename>}.
前面已经看到一个 \texttt{<pkgname>} 可能会对应多个 \texttt{<archivename>},
但基本上 \texttt{<archivename>} 的命名规则是 \texttt{<xxx>},
\texttt{<xxx.source>} 和 \texttt{<xxx.doc>}.
\texttt{<xxx>} 是必装的宏包文件;
\texttt{<xxx.source>} 是选择安装的源码, 如 \texttt{dtx} 文件;
\texttt{<xxx.doc>} 是选择安装的文档, 如 \texttt{pdf} 文件等.
这三者未必同时存在, 例如前面提到的 \texttt{lshort-chinese}.

另外, 如果需要安装的是一个可执行文件, 例如 \texttt{pdftex} 或 \texttt{xetex},
那么会涉及到根据操作系统进行下载的相关判断.
如果要升级本地已安装的可执行文件, 可以在命令行输入以下语句查询:
\begin{lstlisting}[language=bash, title={\small\sffamily Windows 系统}]
  tlmgr info --only-installed | findstr "pdftex"
\end{lstlisting}
%\begin{lstlisting}[title={\small\sffamily Windows 系统}]
%  tlmgr info --only-installed | Select-String -Pattern 'pdftex'
%\end{lstlisting}
\begin{lstlisting}[language=bash, title={\small\sffamily Ubuntu 和 Mac 系统}]
  tlmgr info --only-installed | grep 'pdftex'
\end{lstlisting}
在输出结果中可看到带系统信息的名称.
如果是升级本地未安装的可执行文件,
那么用户就需要根据经验自行判断.

下载正确的 \texttt{<archivename>.tar.xz} 后,
可先检查一下该压缩文件中是否包含了 \texttt{tlpobj} 文件.
之后将压缩文件拷贝到未联网电脑上,
在命令行执行
\begin{lstlisting} 
  tlmgr install --file <archivename>.tar.gz 
\end{lstlisting}
系统便可自行安装.

对于部分提供了 \texttt{l3build} 脚本 \texttt{build.lua} 文件的宏包,
用户可在下载压缩文件后解压,在解压后的目录中执行
\begin{lstlisting} 
  l3build install
\end{lstlisting}
系统便可自行安装.
